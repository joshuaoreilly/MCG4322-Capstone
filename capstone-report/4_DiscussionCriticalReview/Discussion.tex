
Harmonic Drives provide excellent torque in a small form-factor.
Their back-driving torque, however, is relatively low when compared to their output torque \cite{harmonic_drive_csd-2a_2019}.
Springs were added to bring the torque while resting below this limit, at the cost of increasing the width of the entire leg assembly.
Worm gears for transmission may provide a more compact and simple solution, with better resistance to back-driving \cite{spiroidgearing_worm_2019}. It would be beneficial to explore this solution in more depth and compare.

The shafts were made of marine grade stainless steel 316, as it is a common material used in marine applications and is often used for pump shafts \cite{metal_supermarkets_marine_2017}.
Due to the lower yield strength when compared to regular steel, the shaft diameters were too large for the output of the Harmonic Drives. A custom and relatively bulky adapter was required to interface the output bolts of the drive with the shaft.
A stronger grade of steel such as AISI 4140 could be used to reduce the shaft dimension and switch the adapter for a more typical flange collar \cite{mcmaster-carr_rotary_2019}. Since all shafts are contained within the bellows and chassis, the use of stainless steel may not be necessary.

As shown in Appendix \ref{app_sub:linkages}, the method for determining the relative lengths of the leg linkages was fairly experimental and sub-optimal. With more time, the leg linkage ratio optimization could be refined. This might lead to linkage combinations which reduce torques in the legs and match the desired $x$ and $y$ reach better, also likely resulting in a smaller, more compact robot.

The parameterization could also be further developed by including the exact calculations of the normal forces, as well as computing the center of mass and verifying the robot stability at a given slope angle. In this event, maximum terrain slope may have been added as a parameterization input.

As discussed in Section \ref{ssub_sec:constrained_parts}, different motors with higher torques or different mounting dimensions could be explored to avoid interference between the Harmonic Drive and motor when the robot reaches a certain size.

Although the diameter of fasteners was parameterized, they were not included as parts in the final assembly due to time constraints.

Finally, the robot is not very aesthetically pleasing. Considering it will operate in public spaces, a more attractive design could be developed.