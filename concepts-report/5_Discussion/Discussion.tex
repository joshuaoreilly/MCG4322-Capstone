The selected robot solution is concept 2, with the crab-like morphology. Various criteria were used to analyse all concepts and a decision analysis was performed, as detailed in Appendix \ref{app:decision_analysis_tables}. The results for all concepts were close, however it is seen by observing the scoring that concept 2 is the better rounded solution. The feasibility and design complexity criteria is a driving factor in the decision, as the solution for concept 2 is believed to have a better balanced and simplistic approach. For example, although concept 1 seems relatively simple, it is likely to have a large force concentration at the guiding pin. It also has a smaller range of motion and stability. Concept 3 is inherently more complex due to the added motor for the third degree of freedom. It also has a motor on the exterior of the chassis, adding weight (inertia) to the leg and electronics on the outside.

A cost assessment was also performed, as summarized in Table \ref{table:concept_cost_summary}, and is presented in detail in Appendix \ref{app:cost_assessment}. Concept 1 costed around \$16 000, Concept 2 costed around \$27 000 and Concept 3 costed around \$24 000, making Concept 2 the most expensive. For this design, however, cost has very little effect on the decision as the market is not competitive.
Additionally, much of the cost putting it above Concepts 1 and 3 came from the use of carbon-fibre for the linkages (which could be substituted with the cheaper materials used in the other concepts) and Harmonic Drive gearboxes (whose zero-backdrive property may prove beneficial to improving battery life while individual legs are not active, as current will not be required to hold the motor in place). In the latter case, they could be replaced with the much less expensive planetary gearboxes used in Concept 3 if a power consumption analysis shows them to not be worth the additional price.

The chassis and sealing design for the selected concept 2 may be slightly modified and inspired by the chassis designed for the other concepts. It is dependant on further analysis and design of the legs as well as the final location of various components.

The solar panel concepts have also been analysed using a decision analysis table (Appendix \ref{app:decision_analysis_tables}) and a cost assessment (Appendix \ref{app:cost_assessment}). The concept with the highest score is Concept 3, where custom solar panels or solar cells are positioned wherever possible on the top surface of the robot. This solution is retained as the best option for now, but may be changed depending on compatibility with the waste collection system solution and further analysis of power requirements and chassis surface area.

