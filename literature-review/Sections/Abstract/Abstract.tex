Ocean pollution is a significant issue; the goal of this document is to research existing technologies that can be applied in designing a biomimetic robot capable of collecting garbage on beaches.
First, we will present relevant standards and codes to the creation of collaborative or industrial robots.
Then, we will present existing biomimetic robots and relevant robotic devices.
We will break these devices into their relevant subsystems, and offer a detailed look at how they function.
Finally, we will review standard mechanical systems used to create these subsystems and their considerations in a beachfront environment.

In April 2019, Victor Vescovo was the first human to descend to the bottom of the Challenger Deep, the currently known deepest place in the ocean, to discover sea life and plastic waste. 
He claims to have observed a plastic bag and candy wrappers during his four hour journey at the bottom of the ocean \cite{street_deepest_2019}.
The year before, in March 2018, the Scientific Reports published a study detailing the exponential growth of the Great Pacific Garbage Patch.
This island of garbage, is three times the size of France and has accumulated a total of 79,000 tons of plastic, covering 617,800 square miles \cite{abc_great_2018}.