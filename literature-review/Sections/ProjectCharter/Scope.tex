\subsection{Scope}

The team is composed of two separate groups which work together to design a solution. One group (B) is in charge of the garbage collection system, garbage holding tank and associated detection sensors. The other group (A), which is the subject of this report, is responsible for the design of the biomimetic locomotion system, the chassis, the solar power system, the integration of associated electronic controllers and consideration for the integration of the litter removal system.

The device must operate under harsh environments including cold and hot temperatures, rain, salt, water, wind, humidity, mud, rocks and sand. The device's locomotion is also required to operate in arduous topography and low accessibility terrain found near shore lines. This includes sand, pebble beaches, slippery surfaces and obstacles such as rocks, plants, prickly shrubs, grass, uneven landscape and shallow water. No continuously rotating joints, such as wheels, can be used for locomotion.

The device's power systems must be self-powered using solar energy. It must require no human intervention, other than for emptying the collected garbage. It should also be resistant to vandalism. 